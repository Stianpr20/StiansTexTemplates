%----- Abstract algebra and group theory-----%
\newcounter{Teorem}
\newcounter{Korollar}
\newcounter{Proposisjon}
\newcounter{Lemma}

\newcommand{\teorem}{\jump\textbf{{\color{StyleCol3}%
	\iflanguage{norsk}{Teorem\stepcounter{Teorem} \Roman{Teorem}}{}
	\iflanguage{UKenglish}{Theorem\stepcounter{Teorem} \Roman{Teorem}}{}
	}} \\}
	
\newcommand{\Teorem}{\textbf{{\color{StyleCol3}%
	\iflanguage{norsk}{Teorem\stepcounter{Teorem} \Roman{Teorem}}{}
	\iflanguage{UKenglish}{Theorem\stepcounter{Teorem} \Roman{Teorem}}{}
	}} \\}

\newcommand{\teoremnavn}[1]{\jump\textbf{{\color{StyleCol3}%
	\iflanguage{norsk}{Teorem\stepcounter{Teorem} \Roman{Teorem}~--~{#1}}{}
	\iflanguage{UKenglish}{Theorem\stepcounter{Teorem} \Roman{Teorem}~--~{#1}}{}
	}} \\}

\newcommand{\bevis}{\jump\textit{{\color{StyleCol3}%
	\iflanguage{norsk}{Bevis}{}%
	\iflanguage{UKenglish}{Proof}{}%
	}}\\}

\newcommand{\beviS}{\jump\textit{{\color{StyleCol3}%
	\iflanguage{norsk}{Bevis}{}
	\iflanguage{UKenglish}{Proof}{}  }} }

\newcommand{\Bevis}{\textit{{\color{StyleCol3}%
	\iflanguage{norsk}{Bevis}{}
	\iflanguage{UKenglish}{Proof}{}  }} \\}

\newcommand{\lemma}{\jump \textit{\color{StyleCol3} Lemma\stepcounter{Lemma} \Roman{Lemma}} \\}

\newcommand{\Lemma}{\textit{\color{StyleCol3} Lemma\stepcounter{Lemma} \Roman{Lemma}} \\}
	
\newcommand{\korollar}{\jump\textbi{{\color{StyleCol3}%
	\iflanguage{norsk}{Korollar\stepcounter{Korollar} \Roman{Korollar}}{}
	\iflanguage{UKenglish}{Corollary\stepcounter{Korollar} \Roman{Korollar}}{} 
	}} \\}

\newcommand{\korollarnavn}[1]{\jump\textbi{{\color{StyleCol3}%
	\iflanguage{norsk}{Korollar\stepcounter{Korollar} \Roman{Korollar} #1}{}
	\iflanguage{UKenglish}{Corollary\stepcounter{Korollar} \Roman{Korollar} #1}{} 
	}} \\}

\newcommand{\proposisjon}{\jump\textbi{{\color{StyleCol3}%
	\iflanguage{norsk}{Proposisjon\stepcounter{Proposisjon} \Roman{Proposisjon}}{}
	\iflanguage{UKenglish}{Proposition\stepcounter{Proposisjon} \Roman{Proposisjon}}{} 
	}} \\}

\newcommand{\styleQED}{{\color{StyleCol3} $\blacksquare$}}

\newcommand{\mgroup}[2]{\left\langle{#1},\;{#2}\right\rangle}
\newcommand{\mgroupp}[2]{\left({#1},\;{#2}\right)}
\newcommand{\mring}[3]{\left\langle{#1},\,{#2},\,{#3}\right\rangle}
\newcommand{\nn}[1]{$\mathbb{N}_{#1}$}
\newcommand{\zz}[1]{$\mathbb{Z}_{#1}$}
\newcommand{\qq}[1]{$\mathbb{Q}_{#1}$}
\newcommand{\rr}[1]{$\mathbb{R}_{#1}$}
\newcommand{\cc}[1]{$\mathbb{C}_{#1}$}

\newcommand{\deler}[2]{#1 \, \big\vert \, #2}
\newcommand{\Deler}[2]{#1 \, \bigg\vert \, #2}
\newcommand{\delerikke}[2]{#1 \, \slashed{\vert} \, #2}

\newcommand{\polr}[1]{$\mathbb#1\left[x\right]$}
\newcommand{\polrk}{$k\left[x\right]$}
\newcommand{\polrf}{$f\left[x\right]$}
\newcommand{\polrF}{$F\left[x\right]$}
\newcommand{\polrR}{$R\left[x\right]$}
\newcommand{\polrZ}{$\mathbb{Z}\left[x\right]$}
\newcommand{\polrQ}{$\mathbb{Q}\left[x\right]$}
\newcommand{\polrZp}{$\mathbb{Z}_p\left[x\right]$ }
\newcommand{\imbilde}{\mathrm{im}\,\varphi}

\newcommand{\curlydef}[2]{% '{ | }' notation.
	\left\lbrace#1\;\middle|\;#2\right\rbrace}

\newcommand{\injektiv}{\lhook\joinrel\longrightarrow}
\newcommand{\surjektiv}{\relbar\joinrel\twoheadrightarrow}